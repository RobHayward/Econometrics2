\documentclass[12pt]{hitec}\usepackage[]{graphicx}\usepackage[]{color}
%% maxwidth is the original width if it is less than linewidth
%% otherwise use linewidth (to make sure the graphics do not exceed the margin)
\makeatletter
\def\maxwidth{ %
  \ifdim\Gin@nat@width>\linewidth
    \linewidth
  \else
    \Gin@nat@width
  \fi
}
\makeatother

\definecolor{fgcolor}{rgb}{0.345, 0.345, 0.345}
\newcommand{\hlnum}[1]{\textcolor[rgb]{0.686,0.059,0.569}{#1}}%
\newcommand{\hlstr}[1]{\textcolor[rgb]{0.192,0.494,0.8}{#1}}%
\newcommand{\hlcom}[1]{\textcolor[rgb]{0.678,0.584,0.686}{\textit{#1}}}%
\newcommand{\hlopt}[1]{\textcolor[rgb]{0,0,0}{#1}}%
\newcommand{\hlstd}[1]{\textcolor[rgb]{0.345,0.345,0.345}{#1}}%
\newcommand{\hlkwa}[1]{\textcolor[rgb]{0.161,0.373,0.58}{\textbf{#1}}}%
\newcommand{\hlkwb}[1]{\textcolor[rgb]{0.69,0.353,0.396}{#1}}%
\newcommand{\hlkwc}[1]{\textcolor[rgb]{0.333,0.667,0.333}{#1}}%
\newcommand{\hlkwd}[1]{\textcolor[rgb]{0.737,0.353,0.396}{\textbf{#1}}}%

\usepackage{framed}
\makeatletter
\newenvironment{kframe}{%
 \def\at@end@of@kframe{}%
 \ifinner\ifhmode%
  \def\at@end@of@kframe{\end{minipage}}%
  \begin{minipage}{\columnwidth}%
 \fi\fi%
 \def\FrameCommand##1{\hskip\@totalleftmargin \hskip-\fboxsep
 \colorbox{shadecolor}{##1}\hskip-\fboxsep
     % There is no \\@totalrightmargin, so:
     \hskip-\linewidth \hskip-\@totalleftmargin \hskip\columnwidth}%
 \MakeFramed {\advance\hsize-\width
   \@totalleftmargin\z@ \linewidth\hsize
   \@setminipage}}%
 {\par\unskip\endMakeFramed%
 \at@end@of@kframe}
\makeatother

\definecolor{shadecolor}{rgb}{.97, .97, .97}
\definecolor{messagecolor}{rgb}{0, 0, 0}
\definecolor{warningcolor}{rgb}{1, 0, 1}
\definecolor{errorcolor}{rgb}{1, 0, 0}
\newenvironment{knitrout}{}{} % an empty environment to be redefined in TeX

\usepackage{alltt} % Paper size, default font size and one-sided paper
%\graphicspath{{./Figures/}} % Specifies the directory where pictures are stored
%\usepackage[dcucite]{harvard}
\usepackage{amsmath}
\usepackage{setspace}
\usepackage{pdflscape}
\usepackage{rotating}
\usepackage[flushleft]{threeparttable}
\usepackage{multirow}
\usepackage[comma, sort&compress]{natbib}% Use the natbib reference package - read up on this to edit the reference style; if you want text (e.g. Smith et al., 2012) for the in-text references (instead of numbers), remove 'numbers' 
\usepackage{graphicx}
%\bibliographystyle{plainnat}
\bibliographystyle{agsm}
\usepackage[colorlinks = true, citecolor = blue, linkcolor = blue]{hyperref}
%\hypersetup{urlcolor=blue, colorlinks=true} % Colors hyperlinks in blue - change to black if annoying
%\renewcommand[\harvardurl]{URL: \url}
\usepackage{listings}
\usepackage{color}
\definecolor{mygrey}{gray}{0.95}
\lstset{backgroundcolor=\color{mygrey}}
\IfFileExists{upquote.sty}{\usepackage{upquote}}{}
\begin{document}
\title{Introduction to Financial and Capital Markets}
\author{Rob Hayward}
\date{\today}
\maketitle
\section{Introduction}
The financial system does three things:
\begin{itemize}
\item it intermediates between savings and borrowing; 
\item it provides a way to pool,transfer and diversify risk;
\item it runs the payment system
\end{itemize}
% Author: Rasmus Pank Roulund
% Inspired by figure in:
% Howells, Peter og Bain, Keith (2007). Financial markets and
% institutions. 5. udg. Essex: Pearson Education.

\tikzset{
    %Define standard arrow tip
    >=stealth',
    %Define style for boxes
    punkt/.style={
           rectangle,
           rounded corners,
           draw=black, very thick,
           text width=6.5em,
           minimum height=2em,
           text centered},
    % Define arrow style
    pil/.style={
           ->,
           thick,
           shorten <=2pt,
           shorten >=2pt,}
}

\begin{figure}
\begin{tikzpicture}[node distance=1cm, auto,]
 %nodes
 \node[punkt] (market) {Market (b)};
 \node[punkt, inner sep=5pt,below=0.5cm of market]
 (formidler) {Intermediaries (c)};
 % We make a dummy figure to make everything look nice.
 \node[above=of market] (dummy) {};
 \node[right=of dummy] (t) {Ultimate borrower}
   edge[pil,bend left=45] (market.east) % edges are used to connect two nodes
   edge[pil, bend left=45] (formidler.east); % .east since we want
                                             % consistent style
 \node[left=of dummy] (g) {Ultimate lender}
   edge[pil, bend right=45] (market.west)
   edge[pil, bend right=45] (formidler.west)
   edge[pil,<->, bend left=45] node[auto] {Direct (a)} (t);
\end{tikzpicture}

\vspace{1em}

\caption{Elements of the financial system}
\label{figref:finsys}
\end{figure}

%%% Local Variables:
%%% mode: latex
%%% TeX-master: t
%%% End:

\end{document}
